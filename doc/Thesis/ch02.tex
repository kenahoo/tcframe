
\chapter{Background: Text Categorization}
\label{background-tc}

\section{The Text Categorization Process}
\subsection{Document storage}
\label{Document storage}

\subsection{Document format}
\label{Document format}

XXX... However, many Text Categorization data sets are quite large.  They may
commonly be larger than the amount of available memory on the machine
processing them.  This has two important implications.  First,
converting the documents to a special format so that the framework can
access them may be impossible or undesirable for reasons of time,
space, and data redundancy.  Second, a mechanism 

unnecessary copies of the data, 

\subsection{Document structure}
\subsection{Tokenizing of data}
\subsection{Linguistic stemming}
\subsection{Feature selection}
\subsection{Vector space modeling}
\subsection{Machine learning algorithm choice}
\subsection{Machine learning configuration}
\subsection{Incremental or on-line learning}
\subsection{Hypothesis behavior}

\section{Machine learning techniques}
\subsection{Naive Bayes}
\subsection{Decision Trees}
\subsection{Support Vector Machines}

\section{Evaluation}
\subsection{Correctness}
Precision, recall, FN2, error, accuracy
\subsection{Usefulness}
ROC curves
\subsection{Scalability}
\subsubsection{Architecture}
\subsubsection{Running time}
\subsubsection{Memory usage}

\section{Related products}

To establish the relevance of \aicat\ in the marketplace of
Text Categorization, I will examine three related products.  First I
discuss Weka, a Java product that has been used successfully by many
Text Categorization researchers.  Then I address two businesses that
supply products and services related to Text Categorization,
Autonomy.com and Teragram Corporation.

\subsection{Weka}

Weka is an open-source system for Machine Learning originally
developed at the XXX University of Waikato, New Zealand.  Its primary
audience is the international community of academic machine learning
researchers, most notably those working with Categorization or
Clustering problems that arise from working with text.  Weka has
undergone at least one major code rewrite; at present it is
implemented as a set of related Java classes with documented internal
interfaces, so it may itself be considered a framework.

Weka is used extensively throughout the academic Text Categorization
community, and as such includes support for many cutting-edge
categorization techniques, including advances in Support Vector
Machines, k-Nearest-Neighbor, Naive Bayes, and other categorizers, as
well as several variations of feature selection techniques.  It is
therefore a standard against which the \aicat\ framework can
be measured, as well as a resource which can be leveraged in its
construction.

Despite some similar properties, Weka and \aicat\ differ in
their goals and in many important implementation decisions.  Whereas
Weka specifically targets the academic research community,
\aicat\ aims to support use cases under both
application-building and research-conducting situations.
Consequently, Weka will typically keep up with research trends more
closely, but \aicat\ will usually be easier for application
developers to integrate into a real-world situation.

In addition to these differences, another important difference arises
from the different goals in the two projects.  Much of the academic
community is interested in evaluating the correctness and algorithmic
complexity of categorization techniques, whereas most application
developers must also consider resource usage in real-world terms like
time and memory.  In testing, \aicat\ has greatly outperformed
Weka in terms of speed and memory when equivalent algorithms are
compared on identical data sets.  This doesn't reflect an inherent
design flaw in Weka, rather a difference in the kinds of things Weka
developers are likely to spend their time working on.

In order to help facilitate cooperation between the Weka and
\aicat\ communities, as well as leverage existing solutions
inside \aicat, a machine learner class has been created
within \aicat\ that simply passes data through to Weka's
categorizers.  In this way, application developers can easily
experiment with Weka's cutting-edge categorization techniques while
retaining \aicat's application integration advantages.  Any
cross-pollination generated as a result will likely benefit both
projects.

Other facilities provided by Weka are not yet offered by
\aicat.  These include visualization tools, several
sophisticated correctness evaluation tools, and XXX.  Most of these
facilities would make useful additions to \aicat\ if
implemented

\subsection{Autonomy.com}

\subsection{Teragram Corporation}

According to their web site (\url{http://www.teragram.com/}), Teragram
Corporation is a provider of ``fast and stable linguistic
technologies, information search and extraction, knowledge management,
and text processing technologies.''  One of their largest-scale
products is the Teragram Categorizer, an automatic document
categorizer that plays a similar technical role to \aicat.
It cooperates with the Teragram Taxonomy Manager, which provides a
user interface to categories and the documents within each category.

All of Teragram's software products are proprietary, so little
information on implementation is available.  However, product
capabilities and roles can be assessed from the marketing information
given on the web site.  The information presented here has all been
gathered this way.

The Taxonomy Manager is a browser of hierarchical categories, similar
to several on-line directory services like Yahoo
(\url{http://www.yahoo.com/}) or the Open Directory Project
(\url{http://www.dmoz.org/}).  It might therefore be inferred that the
Categorizer is a native hierarchical categorizer, or perhaps that the
categorizer actually flattens the tree structure of the category
hierarchy into a flat list of its leaves, and imposes the tree
structure only afterwards.  Whichever case is true, it must be noted
that the interfaces of the categorizer allow hierarchical
categorization even if the internal workings are flat.

Another interesting aspect of Teragram's categorization technology is
their Rule-Based Categorizer.  Using this system, ``each category
within the directory is associated with a set of rules that describe
documents within that category.''  This may be motivated by a need to
integrate older hand-maintained lists of rules into newer
applications, or it might be meant to address situations like email
categorization in which most documents are indeed best categorized by
simple rules (usually because the sender and receiver have agreed upon
a tagging scheme to mark documents' important properties).  It's not
clear whether Teragram's Rule-Based Categorizer and Automatic
Categorizer can cooperate on a single taxonomy, but they indicate that
the two systems are complementary rather than antithetic.

Teragram also offers separate licensing for many of the tools that
make up its products.  In this sense, it has a strategy similar to one
employed in \aicat's design, in which useful pieces of
functionality created for \aicat\ should be split off into
their own products whenever possible.

