\chapter{Conclusion}
\label{Conclusion}

This thesis has presented the background, design, implementation, and
evaluation of a new object-oriented application framework for Text
Categorization.  The framework does not include any novel work in Text
Categorization algorithms, but it has been designed with the intention
of facilitating such novel work.

The evaluation in Chapter \ref{Evaluation} shows that the \naive\
Bayes implementation gives results consistent with the work of others
in the TC literature, and that the framework's k-Nearest-Neighbor and
Support Vector Machine Implementations are not yet comparable with the
state-of-the-art implementations of the algorithms referenced in the
literature.

The architectural quality of the framework is difficult to evaluate
directly, but several analysis approaches have been presented in this
thesis.  Several aspects of the framework's design have been presented
as instances of Design Patterns, indicating that they may align with
current best-practices in framework development.  Example applications
of several types have also been discussed, indicating that the
framework can support the use cases presented in Section
\ref{use-cases}.

\section{Further Work}

Since one of the main goals of framework development is that the
framework should be easily extensible and developers can add new
functionality with a minimum of effort, there are many avenues for
further work with the framework.  For instance, adding common
functionality in the core framework is helpful as the framework
progresses from a whitebox to a more blackbox style of usage.

A selection of some of the most desirable directions for further work
is presented here, in no particular order.

\begin{itemize}
\item As mentioned in Section \ref{imp-featurevectors}, different
  feature vector data structures may be desirable in different
  situations.  To provide framework users with the flexibility to make
  trade-offs between speed, memory, and complexity, the sparse
  \texttt{C}-level structure described there should be implemented.
\item Many new algorithms are continually being developed and refined
  in the TC literature, and the most promising and general-purpose of
  these should be incorporated into the framework.
\item In order to allow application developers to use and extend the
  framework more easily, different kinds of documentation, such as
  tutorials and recipes, should be written.  \cite{fayad:99} suggests
  that multiple kinds of documentation can aid framework adoption.
\item The current Support Vector Machine and k-Nearest-Neighbor
  categorizers should be investigated and improved so that they
  deliver performance matching the published results in the well-known
  TC literature.
\item It might be helpful if the functionality in the \naive\ Bayes
  categorizer were available as a separate module outside the
  framework, so that developers interested in Machine Learning
  problems outside the TC domain could apply the algorithms it
  implements.  The implementation in \aicat\ could then become a
  simple Adapter to the outside module.
\item The framework should be put to use in a hierarchical
  categorization application, with possible hierarchical functionality
  added to the framework to support such usage.
\item Alternate feature selection algorithms, such as the $\chi^2$ and
  $IG$ algorithms discussed in Section \ref{dim-reduction}, should be
  implemented and added to the core framework.
\end{itemize}
