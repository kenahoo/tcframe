\chapter{Abstract}

The field of automatic Text Categorization (TC) concerns the creation
of categorizer functions which assign categories from a pre-defined
set of labels to documents based on the documents' content.  The most
successful modern approaches to TC employ Machine Learning to create
these categorizer functions automatically from a training set of
documents similar to the ones in the target application.

Techniques for creating the categorizer functions are the subject of
active research in the Machine Learning and Text Processing academic
communities, and every stage of the process has many variations.
Because of this complexity, the creation of TC applications is usually
a complicated process requiring a great deal of expertise in TC, the
application domain, and software development.

This thesis concerns the creation of an Object-Oriented Application
Framework for Text Categorization.  Frameworks provide a way to
address difficult software engineering problems by encapsulating the
aspects of applications that tend to remain common from one
application to the next, while allowing variation in the aspects that
tend to change.  This allows significant code and design reuse when
building new applications.

Framework design is among the most difficult tasks in software
engineering, and requires a great deal of problem analysis,
architecture design, and product evaluation.  Design Patterns have
grown popular in order to manage the complexities of these aspects and
to facilitate communication among those involved with the
Object-Oriented design of large software systems.  This thesis uses
Design Patterns to examine several of the important internal
relationships among framework classes.

XXX (Will write this last)

\chapter*{Acknowledgments}

I would like to thank Rafael Calvo for his expert supervision of my
thesis, and for giving me the opportunity to pursue this project.  The
rest of the Web Engineering Group---Jae-Moon Lee, Xiaobo Li, Nick
Carroll, and Gosia Mandrela---provided valuable testing and feedback
on \aicat\ and created a quite pleasant research environment.  The
Language Technology group---particularly Casey Whitelaw, Elisabeth
Crawford, and Jon Patrick---was also a good source of feedback and
inspiration.  The Open-Source community provided incentives to write
clean, usable, documented software by their mere existence.  Research
Assistantship funding from the Capital Markets Co-operative Research
Centre provided much-appreciated support for research on the Signal G
corpus.

On a more personal level, I would like Sheri Schechinger for sticking
with me across ten thousand miles of ocean, and the city of Sydney for
being such a great place to spend time.

\chapter*{Preface}

XXX - Rafael says this is too technical in the beginning.

This thesis presents research into design and implementation of a framework
for automatic Text Categorization (TC).  In order to produce such a
framework, research into current TC algorithms has been necessary, as
well as research into software engineering practices for building
object-oriented frameworks.

The framework, called \aicat, has been designed with the goal of being
generically useful for building real-world TC applications, and for
being extensible using common framework extension techniques.

\section*{Availability}

The latest released version of the \aicat\ framework (currently 0.04)
is always available at
\url{http://search.cpan.org/author/KWILLIAMS/}.  Perl source code,
documentation, and a simple example application are included in the
distribution.

For developers who wish to stay more actively involved with tracking
changes in the framework, the entire distribution is also available
using the Concurrent Versions System (CVS).  This allows developers to
access the latest bug fixes, to create their own patches against the
main framework code, and to track changes between releases.  Details
of how to access the CVS version are at
\url{http://sourceforge.net/cvs/?group_id=62831}, or via the project's
development home page at
\url{http://sourceforge.net/projects/ai-categorizer/}.

The ApteMod data set discussed in Chapter \ref{Evaluation} is
available for download from
\url{http://kdd.ics.uci.edu/databases/reuters21578/reuters21578.html}.

The Dr. Math data set is not available for direct download, but
interested parties may contact Ken Williams at \url{ken@mathforum.org}
for details.

XXX - Signal G data set needs to be discussed if it remains in the
thesis

After submission to the University of Sydney, this thesis document
will be available in electronic format at
\url{http://www.ee.usyd.edu.au/~kenw/Thesis.pdf}, and in hardcopy
format from the University of Sydney Engineering Library.

\section*{Licensing}

The \aicat\ framework is implemented as a set of Perl modules (see
Section \ref{imp-language}).  As is customary with many Perl modules,
the framework is distributed under the same licensing terms as the
standard version of the Perl interpreter.  This means that the user
may choose either the GNU General Public License or the Artistic
License as the terms of using the software, whichever fits better with
their needs.  In practical terms, this means that the code is
encouraged to be used in research, commercial, educational, or other
environments, without the need to pay royalties to the software's
original author.  It also means that the software's inner workings are
available to be inspected or modified by other developers for their
own projects.

Licenses of the above type are called ``open source'' licenses.  Their
goal is to foster the development and evolution of software by
leveraging the user community and developer community as a resource
that can feed back into the development cycle.  According to
\url{http://www.opensource.org/}, ``open source promotes software
reliability and quality by supporting independent peer review and
rapid evolution of source code.''  This aligns very well with the
traditional goals of academic research.  By making the source code
discussed in academic project available as open source resources, the
results can far more easily be verified by other researchers.

For more information on open source concepts, please visit
\url{http://www.opensource.org/}.

This thesis is copyright \copyright 2003 by Ken Williams. This
material may be distributed only subject to the terms and conditions
set forth in the Open Publication License, v1.0 or later (the latest
version is presently available at
\url{http://www.opencontent.org/openpub/}).

\section*{Motivations}

My own personal motivations for embarking on this project were to
further educate myself on Text Categorization research, to learn more
about framework methodology, and to provide a software resource to
others who wish to use TC methods in their software projects.  I have
long been interested in Machine Learning methods for various purposes,
and I enjoy working with natural languages.  Doing corpus-based work
in Natural Language Processing is a fun combination of Machine
Learning and Linguistics, and reading the literature on the topic
always makes me excited to work on my next project.

Unfortunately, when I was just beginning to do work on my own Text
Categorization projects, I found that there were very few TC tools
freely available for my use, and those that were available were often
difficult to customize.  Very few tools were available in Perl, my
usual language of choice, and this seemed like an odd situation given
the well-known agility of Perl at handling text data.  I cannot hope
to solve everyone's software needs in TC, but the \aicat\ framework
represents my best effort in providing the kind of thing I was looking
for when I began working in this area.

My interest in framework development has recently increased by working
on the \class{HTML::Mason} project \cite{rolsky:02}.  For this, I
and others helped shepherd the code from a fairly monolithic
function-based tool to a customizable OO framework suitable for many
more purposes than it was originally developed to serve.  I became
convinced of the power of framework development with that project, and
I sought to bring the same benefits to an open-source framework for TC.


\section*{Contributions}

During the course of the candidature on which this thesis is based,
the following contributions were accomplished:

\begin{itemize}
\item The \aicat\ framework was designed, implemented, and released
  under an open-source license \cite{cpan}.  The release includes
  documentation and a simple example application using the framework.
\item \naive\ Bayes and Decision Tree categorizers were implemented,
  as well as a mechanism which allows users to use categorizers
  implemented in the Weka Machine Learning system\cite{weka:99}.  A
  simple probabilistic guessing categorizer has also been implemented
  to provide a baseline for experimentation.
\item Contributions from other developers have provided the framework
  with an SVM categorizer.  Collaborative work with other developers
  have provided Rocchio and k-Nearest-Neighbor categorizers.
\item The framework currently has a Document Frequency feature
  selection module implemented.
\item A paper on the design and applicability of the \aicat\ framework
  was published in the proceedings of the 7th Australasian Document
  Computing Symposium. \cite{williams:02}
\item A short paper on the use of the \aicat\ framework to categorize
  financial documents was published in the proceedings of the
  7th Australasian Document Computing Symposium. \cite{calvo:02}
\item A paper on the use of \aicat\ to automatically categorize
  mathematics questions will be published in the
  proceedings of the 11th International Conference on Artificial Intelligence in
  Education.  \cite{williams:03}
\item An overview seminar on TC and the design of \aicat\ was given at
  the University of Sydney.  An invited presentation of the same
  seminar was given to the Language Technology group at Macquarie
  University.
\item Tutorials on Machine Learning were presented at the O'Reilly
  2002 Open Source Conference and 2003 Bioinformatics Technology
  Conference (\url{http://conferences.oreilly.com/}).
\item New testing corpora have been assembled in the educational and
  financial domains, and the framework has been evaluated using them
  (see Chapter \ref{Evaluation}).
\end{itemize}

\section*{Organization of the Thesis}

Chapter \ref{background-tc} gives a detailed account of the main technical
issues in Text Categorization that a TC framework must take into
account.  Chapter \ref{design} discusses design issues in creating the
\aicat\ framework, motivating the design by consideration of the
framework's audience and common usage scenarios, and showing some of
the limitations of the framework's design.  Chapter
\ref{Implementation} is a short discussion of implementation issues.
Chapter \ref{Evaluation} evaluates the framework from several
different perspectives, and Chapter \ref{Conclusion} concludes.
