\chapter{Introduction}

\section{Preface}

This thesis represents research into and implementation of a framework
for automatic Text Categorization (TC).  In order to produce such a
framework, research into current TC algorithms has been necessary, as
well as research into software engineering practices for building
object-oriented frameworks.

The framework, called \aicat, has been designed with the goal of being
generically useful for building real-world TC applications, and for
being extensible using common framework extension techniques.

During the course of the candidature on which this thesis is based,
two papers co-authored by Ken Williams were
published. \cite{williams:02, calvo:02} Another paper is currently
under consideration for publication.  \cite{williams:03} Seminars were
given on the current research at The University of Sydney and
Macquarie University.  The code and documentation for the \aicat\
framework was published with several cycles of releases. \cite{cpan}
This thesis draws upon all this work for its results.

\section{Automatic Text Categorization}
\label{tc-intro}

The field of automatic Text Categorization is an extremely active area
of current research and application.  It is a multi-disciplinary
field, attracting attention from the Linguistics, Computer Science,
Engineering, and Business communities.  Its applicability is broad,
with many potential uses for large businesses as well as individuals.
A recent survey article from the Association of Computing Machinery
provides a good introduction to the field.\cite{sebastiani:02}

The goal of automatic Text Categorization is to automatically produce specialized
algorithms that can process natural-language documents, assigning zero
or more user-defined labels to them based on their content.  More
formally, given a set of labels (i.e., categories) $\cats = \{c_1, \ldots, c_{|\cats|}\}$ and a set of
previously unseen documents $\docs = \{d_1, d_2, \ldots \}$, a categorizer is a
function that maps from $\docs$ to the set of all subsets of $\cats$.  In
some applications, categorizers assign only a single label to each
document, so a categorizer is often a function that maps directly from
$\docs$ to $\cats$.  Often an intermediate function is useful for ``soft'' or 
``rank-based'' categorization, mapping from $\docs \times \cats$ to
the set of real numbers $\mathbb{R}$ in order to assign a score to
each category $c_j$ for each document $d_i$.  The scored categories
may then be presented to a human expert in decreasing order, and the
human may then make the final decision on the document's category
membership.

The standard modern approach to creating new categorizer functions is
to build them using Machine Learning techniques from a set of training
documents $\train$.\cite[p. 2]{sebastiani:02} This is a set of
user-provided, pre-labeled documents that follows a category
distribution similar to the distribution of $\docs$, and whose contents
provide information about what sorts of documents should be mapped to
what sorts of categories.  Algorithms can then be developed that make
generalizations about the relationship between document content and
document category, encoding these generalizations in the learned algorithm.

\section{Object Frameworks}

A framework is a large-scale unit of reusable code in object-oriented
software development.  Frameworks were developed in response to
situations requiring fine-grained control over the

\subsection{Guidelines for designing frameworks}

XXX - needs to be written, distilled from \cite{fayad:99}

\subsection{Design patterns}

XXX - needs to be written

\section{Contributions}

XXX - list what I feel are the research contributions of this thesis,
and say what publications have arisen from this work

\section{Organization of the Thesis}

Chapter \ref{background-tc} gives a detailed account of the technical
issues in Text Categorization that a TC framework must take into
account.  Chapter \ref{design} discusses design issues in creating the
\aicat\ framework, motivating the design by consideration of the
framework's audience and common usage scenarios.  Chapter
\ref{Implementation} is a short discussion of implementation issues.
Chapter \ref{Evaluation} evaluates the framework from several
different perspectives, and Chapter \ref{Conclusion} concludes.
