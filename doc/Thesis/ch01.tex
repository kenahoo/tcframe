\chapter{Introduction}

\section{Preface}

The field of Automatic Text Categorization is an extremely active area
of current research and application.  It is a multi-disciplinary
field, attracting attention from the Linguistics, Computer Science,
Engineering, and Business communities.  Its applicability is broad,
with many potential uses for large businesses as well as individuals.

The goal of Automatic Text Categorization is to produce specialized
algorithms that can process natural-language documents, assigning zero
or more user-defined labels to them based on their content.  More
formally, given a set of labels $L = \{L_1, \ldots, L_k\}$ and a set of
previously unseen documents $D = \{D_1, D_2, \ldots \}$, a categorizer is a
function $C$ that maps from $D$ to the set of all subsets of $L$.  In
practice, many categorizers assign only a single label to each
document, so a categorizer is often a function that maps directly from
$D$ to $L$.  Often an intermediate function is useful for ``soft'' or 
``rank-based'' categorization, mapping from ordered pairs $(D_i, C_j)$ 
to the set of real numbers $\mathbb{R}$ in order to assign a score 
to each category $C_j$ for each document $D_i$.

The standard modern approach to creating new categorizer functions is
to build them using Machine Learning techniques from a set of training
documents.  This is a set of user-provided, pre-labeled documents that
follows a category
distribution similar to the distribution of $D$, and whose contents
provide information about what sorts of documents should be mapped to
what sorts of categories.  Algorithms can then be developed that make
generalizations about the relationship between document content and
document category, encoding these generalizations in the algorithm $C$.

\section{Automatic Text Categorization}

\section{Object Frameworks}

A framework is a large-scale unit of reusable code in object-oriented
software development.  Frameworks were developed in response to
situations requiring fine-grained control over the

\subsection{Guidelines for designing frameworks}
\subsection{Design patterns}


\section{Contributions}

\section{Organization of the Thesis}

XXX  - can briefly summarize contents of other chapters