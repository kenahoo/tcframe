\chapter{Introduction}

\section{Automatic Text Categorization}
\label{tc-intro}

XXX - need to have more of a summary about why TC and IR are important
(from Rafael: ``start with an overview of why TC and IR are important,
in layman terms.  You need to sell the need for your work, you can not
start with the description of vector models. You can mention
information overload, the need for knowledge management, document and
email filetering and routing, etc'')

The field of automatic Text Categorization is an extremely active area
of current research and application.  It is a multi-disciplinary
field, attracting attention from the Linguistics, Computer Science,
Engineering, and Business communities.  Its applicability is broad,
with many potential uses for large businesses as well as individuals.
A recent survey article from the Association of Computing Machinery
provides a good introduction to the field.\cite{sebastiani:02}

The goal of automatic Text Categorization is to automatically produce specialized
algorithms that can process natural-language documents, assigning zero
or more user-defined labels to them based on their content.  More
formally, given a set of labels (i.e., categories) $\cats = \{c_1, \ldots, c_{|\cats|}\}$ and a set of
previously unseen documents $\docs = \{d_1, d_2, \ldots \}$, a categorizer is a
function that maps from $\docs$ to the set of all subsets of $\cats$.

XXX - create a diagram of a classifier's action on sets

In
some applications, categorizers assign only a single label to each
document, so a categorizer is often a function that maps directly from
$\docs$ to $\cats$.  Often an intermediate function is useful for ``soft'' or 
``rank-based'' categorization, mapping from $\docs \times \cats$ to
the set of real numbers $\mathbb{R}$ in order to assign a score to
each category $c_j$ for each document $d_i$.  The scored categories
may then be presented to a human expert in decreasing order, and the
human may then make the final decision on the document's category
membership.

The standard modern approach to creating new categorizer functions is
to build them using Machine Learning techniques from a set of training
documents $\train$.\cite[p. 2]{sebastiani:02} This is a set of
user-provided, pre-labeled documents that follows a category
distribution similar to the distribution of $\docs$, and whose contents
provide information about what sorts of documents should be mapped to
what sorts of categories.  Algorithms can then be developed that make
generalizations about the relationship between document content and
document category, encoding these generalizations in the learned algorithm.

\section{Object-Oriented Application Frameworks}

An Object-Oriented Application Framework (hereafter referred to simply
as a ``framework'') is a large-scale unit of reusable code in
object-oriented
software development.  Frameworks were developed in response to
situations requiring fine-grained control over the \ldots XXX \ldots

\subsection{Guidelines for designing frameworks}

XXX - needs to be written, distilled from \cite{fayad:99}

 * white-box vs. black-box needs to be defined

\subsection{Design patterns}
\label{patterns}

In order to shed light on the design of complex object-oriented
systems, many researchers and software developers have tried to
standardize language, concepts, and notation for class and object
relationships.  There is as yet no universally accepted terminology
for describing these relationships, but one common practice is to use
so-called ``design patterns'' to provide a baseline grammar for
discussing commonly seen patterns of cooperation in object-oriented
design. \cite[p. 3]{gamma:95} The design patterns do not provide
prescriptions for software design, but rather descriptions of common
practices in common situations.  Most design patterns in
\cite{gamma:95} include discussions of various trade-offs in their
application, indicating that a design pattern is actually a family of
similar solutions to a problem, not one rigid solution.

Design patterns help to illustrate object-oriented software designs
that use \emph{composition} rather than just \emph{inheritance} for
embodying important relationships between objects.  Composition refers
to the practice of multiple independent objects cooperating to achieve
a task, or assembling to form a larger functional unit, while
inheritance refers to the practice of defining a single object's
structure and behavior in terms of both general (``parent'') and
specific (``child'') specifications.  In the language of framework
design and reuse, composition allows for black-box reuse, while pure
inheritance forces white-box reuse.\cite[p. 19]{gamma:95}

The relevance of several design patterns to \aicat\ will be discussed
in detail in Chapter \ref{design}.

\section{Contributions}

During the course of the candidature on which this thesis is based,
the following contributions were accomplished:

\begin{itemize}
\item The \aicat\ framework was designed, implemented, and released
  under an open-source license \cite{cpan}.  The release includes
  documentation and a simple example application using the framework.
\item \naive\ Bayes and Decision Tree categorizers were implemented,
  as well as a mechanism which allows users to use categorizers
  implemented in the Weka Machine Learning system\cite{weka:99}.  A
  simple probabilistic guessing categorizer has also been implemented
  to provide a baseline for experimentation.
\item Contributions from other developers have provided the framework
  with an SVM categorizer.  Collaborative work with other developers
  have provided Rocchio and k-Nearest-Neighbor categorizers.
\item The framework currently has a Document Frequency feature
  selection module implemented.
\item A paper on the design and applicability of the \aicat\ framework
  was published in the proceedings of the Australian Document
  Computing Symposium. \cite{williams:02}
\item A short paper on the use of the \aicat\ framework to categorize
  financial documents was published in the proceedings of the
  Australian Document Computing Symposium. \cite{calvo:02}
\item A paper on the use of \aicat\ to automatically categorize
  mathematics questions is under consideration for publication in the
  proceedings of the conference on Artificial Intelligence in
  Education.  \cite{williams:03}
\item An overview seminar on TC and the design of \aicat\ was given at
  the University of Sydney.  An invited presentation of the same
  seminar was given to the Language Technology group at Macquarie
  University.
\item Tutorials on Machine Learning were presented at the O'Reilly
  2002 Open Source Conference and 2003 Bioinformatics Technology
  Conference (\url{http://conferences.oreilly.com/}).
\item New testing corpora have been assembled in the educational and
  financial domains, and the framework has been evaluated using them
  (see Chapter \ref{Evaluation}).
\end{itemize}


\section{Organization of the Thesis}

Chapter \ref{background-tc} gives a detailed account of the technical
issues in Text Categorization that a TC framework must take into
account.  Chapter \ref{design} discusses design issues in creating the
\aicat\ framework, motivating the design by consideration of the
framework's audience and common usage scenarios, and showing some of
the limitations of the framework's design.  Chapter
\ref{Implementation} is a short discussion of implementation issues.
Chapter \ref{Evaluation} evaluates the framework from several
different perspectives, and Chapter \ref{Conclusion} concludes.
